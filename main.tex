\documentclass{article}
\usepackage{graphicx}
\usepackage{multirow}
\usepackage{caption}
\usepackage[utf8]{inputenc}
  \usepackage[
    backend=biber,
    style=ieee,
  ]{biblatex}
 


\title{CS6000 - Seventh Assignment}
\author{Marc Moreno Lopez}
\date{October 15th 2018}

\begin{document}

\maketitle

\section{Report}

%Describe your learning/process so far on the survey paper.  I expect this to be short. 

During this week I have been working towards the completion of the second draft for the survey paper. This version of the paper is a really close version to the final one. Bill and I might will probably keep working on it during this week, editing as much as we can and trying to correct any grammar mistakes.

One of the things that I struggled the most with was the unsupervised learning part. It took me a while to rewrite everything like a story, specially the Biometrics section. Since English isn't my first language, sometimes it can be hard to rewrite something and make it look different without changing the meaning. I struggled with the Biometrics section because it was the section where the papers had less in common and I couldn't come up with a storyline. 

Another thing that I struggled with was the conclusions section. For me the most difficult part to write for any paper are the conclusions. I have always struggled with the, specially with survey papers. When I'm working on a research paper, it's my own work, so I know exactly what I want to get from that paper and what I want the reader to keep from my paper. In this case it's a bit more difficult. So I'm looking forward to read any comments on the conclusions from the reviewer. I might also ask Dr. Boult about it. We also decided to change the abstract and will probably keep changing it until we are both happy. I'm also looking forward to any comments form the reviewers about it.

Finally, we decided to use images from the papers. Bill and I decided to use one or two figure examples in each category to show how an algorithm that belongs to that section might look like. 

Inside my repo I have cloned the overleaf project. In this folder you can find all the files for the tex file.




\end{document}
